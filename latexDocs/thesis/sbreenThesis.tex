\documentclass[12pt,Bold,letterpaper]{mcgilletdclass}
\usepackage[dvips,final]{graphicx}
\usepackage[dvips]{geometry}
\usepackage{float}		% this is to place figures where requested!
\usepackage{times}		% this uses fonts which will look nice in PDF
\usepackage{graphicx}		% needed for the figures
\usepackage{epstopdf}
\usepackage{url}
\usepackage{amsfonts }
\usepackage{amssymb,amsmath}
\usepackage{algorithm}
\usepackage{algorithmic}
%\usepackage{program}
\usepackage{amsthm }
\usepackage{amsmath}
\usepackage{tikz}

\newcommand{\bmx}{\begin{bmatrix}}
\newcommand{\emx}{\end{bmatrix}}

% Equations:
\newcommand{\be}{\begin{equation}}
\newcommand{\ee}{\end{equation}}
\newcommand{\beqy}{\begin{eqnarray}}
\newcommand{\eeqy}{\end{eqnarray}}
\newcommand{\beqynn}{\begin{eqnarray*}}
\newcommand{\eeqynn}{\end{eqnarray*}}
\newcommand{\boxcon}{{\cal B}}
%\usepackage{tex4ht}
%\usepackage{amsmath}
%%%%%%%%%%%%%%%%%%%%%%%%%%%%%%%%%%%%%%%%%%%%%%%%%%%%%
%% Have you configured your TeX system for proper  %%
%% page alignment? See the McGillETD documentation %%
%% for two methods that can be used to control     %%
%% page alignment. One method is demonstrated      %%
%% below. See documentation and the ufalign.tex    %%
%% file for instructions on how to adjust these    %%
%% parameters.                                     %%
\addtolength{\hoffset}{0pt}                        %%
\addtolength{\voffset}{0pt}                        %%
%%                                                 %%
%%%%%%%%%%%%%%%%%%%%%%%%%%%%%%%%%%%%%%%%%%%%%%%%%%%%%
%%       Define student-specific info
\SetTitle{\huge{Integer Least Squares\\Search and Reduction Strategies}}%
\SetAuthor{Stephen Breen}%
\SetDegreeType{Master of Science}%
\SetDepartment{School of Computer Science}%
\SetUniversity{McGill University}%
\SetUniversityAddr{Montreal,Quebec}%
\SetThesisDate{August, 2011}%
\SetRequirements{A thesis submitted to McGill University\\
in partial fulfilment of the requirements of the degree of\\
Master of Science in Computer Science}%
\SetCopyright{\copyright Stephen Breen 2011}%

\makeindex[keylist]
\makeindex[abbr]

%% Input any special commands below
%\newcommand{\Kron}[1]{\ensuremath{\delta_{K}\left(#1\right)}}
\listfiles%
\begin{document}

\maketitle%

\begin{romanPagenumber}{2}%

\SetDedicationName{\MakeUppercase{Dedication}}%
\SetDedicationText{}%
\Dedication%

\SetAcknowledgeName{\MakeUppercase{Acknowledgements}}%
\SetAcknowledgeText{}%
\Acknowledge%


%%%%%%%%%%%%%%%%%%%%%%%%%%%%%%%%%%%%%%%%%%%%%%%%%%%%%
%%         English Abstract                        %%
%%%%%%%%%%%%%%%%%%%%%%%%%%%%%%%%%%%%%%%%%%%%%%%%%%%%%
\SetAbstractEnName{\MakeUppercase{Abstract}}%
\SetAbstractEnText{In the worst case the integer least squares (ILS) problem is
NP-Hard. Since its solution has many practical applications, there have been a
number of algorithms proposed to solve it and some of its variations e.g., the
box-constrained ILS problem (BILS). There are typically two stages to solving an
ILS problem, the reduction and the search. Obviously we would like to solve
instances of the ILS problem as quickly as possible, however most of the
literature does not compare the run-time or FLOP counts of the algorithms,
instead they use a more abstract metric (the number of nodes explored during the
search). This metric does not always co-incide with the algorithms run-time.
This thesis will review some of the most effective reduction and search
strategies for both the ILS and BILS problems. By comparing the run-time
performance of some search algorithms, we are able to see the advantages of
each, which allows us to propose a new, more efficent search strategy that is a
combination of two others. We also prove that two very effective BILS reduction
strategies are theoretically equivalent and propose a new BILS reduction that
is equivalent to the others but more efficeint.}
\AbstractEn%

%%%%%%%%%%%%%%%%%%%%%%%%%%%%%%%%%%%%%%%%%%%%%%%%%%%%%
%%         French Abstract                         %%
%%%%%%%%%%%%%%%%%%%%%%%%%%%%%%%%%%%%%%%%%%%%%%%%%%%%%
\SetAbstractFrName{\MakeUppercase{ABR\'{E}G\'{E}}}%
\SetAbstractFrText{ The text of the abstract in French begins here.  }%
\AbstractFr%

\TOCHeading{\MakeUppercase{Table of Contents}}%
\LOTHeading{\MakeUppercase{List of Tables}}%
\LOFHeading{\MakeUppercase{List of Figures}}%
\tableofcontents %
\listoftables %
\listoffigures %

\end{romanPagenumber}

%\mainmatter %
 
\chapter{Introduction}
\section{Least Squares Problem}
Consider the following linear model for some observation vector $y$,
\begin{equation}
\label{eq:realLSModel}
y = Ax+v.
\end{equation}

Where $y\in\mathbb{R}^m$, $A\in\mathbb{R}^{m \times n}$ is called the
``design matrix'' and has full column rank, and $v\in\mathbb{R}^m$ is a
noise vector which we assume is normally distributed with mean $0$ and
covariance matrix $\sigma^2I$. We would like to find the unique solution
$x\in\mathbb{R}^m$ which minimizes the least squares residual,
\be
\label{eq:realLSResidual}
 \left \| Ax - y \right \|^2_2.
\ee

This is called the least squares (LS) problem. If we expand
\eqref{eq:realLSResidual} and set its gradient to $0$, we will arrive at the
well known ``normal equations'' which can be written in matrix form as,
\begin{align}
&A^TAx = A^Ty \\
\label{eq:normalEquations}
&x = (A^TA)^{-1}A^Ty.
\end{align}

The solution of the least squares problem has numerous applications in almost
every field of science and engineering.

\section{Integer Least Squares Problems}
The integer least squres (ILS) problem is a modification of the LS problem
where the solution vector $x\in\mathbb{Z}^m$ is an unknown integer vector. We
no longer have a closed-form solution for $x$ in this case, in fact, the
problem is provably NP-Hard in the worst case.

A modification to the ILS problem is the box-constrained integer least squares
problem (BILS). Here we have the following constraint on the solution
vector, 
\begin{align}
&x\in \boxcon\\
&\boxcon = \{x\in\mathbb{Z}^n : l\le x \le u, l\in \mathbb{Z}^n,
u\in\mathbb{Z}^n \}.
\end{align}

Some important applications such as MIMO wireless signal decoding depend on the
solution of the BILS problem. MIMO stands for ``multiple-input
multiple-output'', it refers to the case where a wireless system has multiple
input antennas transmitting a signal which is received by multiple output
antennas. The signal received is our input vector $y$ from
\eqref{eq:realLSModel}, it has undergone some linear transformation by the known
``channel matrix'' $A$ (design matrix) and some noise has been introduced during
the transmission. Originally, we know that each element of $x$ came from some
finite set of symbols that we may want to transmit or receive (we model
this property with \boxcon). The purpose of such a system is to maximize
throughput, however, the overall throughput of the system will depend on how
quickly we can solve the BILS problem. Of course we need not solve the BILS
problem exactly, but under the assumption that the noise has $0$ mean and is
normally distributed, the BILS solution is more likely than any other possible
solution to be the true solution $x$.

Other applications of BILS and ILS include global positioning systems,
cryptography, lattice design, etc... Any application where the elements of $x$
are known to be integer, we should use ILS. If the elements of $x$ are drawn
from some finite set, BILS is appropriate.

Even though the problem is NP-Hard, we still have some hope to get solutions
quickly. In \cite{HasV05} the authors prove that under reasonable assumptions
on the variance in the noise, the ILS problem can be solved in polynomial time
using standard search algorithms.

The usual approach to solving an ILS or BILS problem consists of two phases,
reduction and search. In the reduction phase, we transform the problem into an
equivalent, but easier one. This involves manipulations such as permutations
and possibly integer gauss transformations on the design matrix $A$. After
reduction, we proceed to the search phase where we try to enumerate the possible
solutions in an efficient manner.

\chapter{Search Algorithms}
There are many search algorithms that have been proposed to solve the ILS
problem. This chapter will review some of the most effective search algorithms
in the literature for both the ILS and BILS problems.

\section{Schnorr-Euchner Enumeration}
The Schnorr-Euchner (SE) enumeration strategy is the 

\bibHeading{References}
\bibliography{../ILS}
\bibliographystyle{plain}

\index[abbr]{LS@LS: Least Squares}
\index[abbr]{ILS@ILS: Integer Least Squares}
\index[abbr]{BILS@BILS: Box-constrained Integer Least Squares}

\printindex[keylist]{Index}{Index}{}
\printindex[abbr]{KEY TO ABBREVIATIONS}{KEY TO ABBREVIATIONS}{}

\end{document}


 






